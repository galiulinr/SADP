\documentclass[a4paper,12pt]{article}

% Подключение необходимых пакетов
\usepackage[left=30mm,right=10mm,top=2cm,bottom=2cm]{geometry} % Поля 
\usepackage{amsmath,amsthm,amssymb}
\usepackage{mathtext}
\usepackage[T1,T2A]{fontenc}
\usepackage[utf8]{inputenc}
\usepackage[english,russian]{babel}
\usepackage{graphicx}
\usepackage{tabularx} % Пакет для таблиц с автоматической шириной столбцов

\usepackage{setspace} % Межстрочные интервалы
\usepackage{indentfirst} % Отступ у первого абзаца
\usepackage{titlesec} % Настройка заголовков
\usepackage{csquotes} % Для цитат
\usepackage{hyperref} % Для гиперссылок
\usepackage{amsmath}
\usepackage{amssymb}
\usepackage{xypic}
\usepackage{tikz}

\usepackage{listings}
\lstset{inputencoding=utf8x, extendedchars=\true}

% Define a custom color
\definecolor{backcolour}{rgb}{0.95,0.95,0.92}
\definecolor{codegreen}{rgb}{0,0.6,0}

% Define a custom style
\lstdefinestyle{myStyle}{
	backgroundcolor=\color{backcolour},   
	commentstyle=\color{codegreen},
	basicstyle=\ttfamily\footnotesize,
	breakatwhitespace=false,         
	breaklines=true,                 
	keepspaces=true,                 
	numbers=left,       
	numbersep=5pt,                  
	showspaces=false,                
	showstringspaces=false,
	showtabs=false,                  
	tabsize=2,
	basicstyle=\footnotesize\ttfamily,
	keywordstyle=\bfseries\color{green!40!black},
	commentstyle=\itshape\color{purple!40!black},
	identifierstyle=\color{blue},
	backgroundcolor=\color{gray!10!white},
}

% Use \lstset to make myStyle the global default
\lstset{style=myStyle}


\usetikzlibrary{positioning}

% Форматирование заголовков разделов
\titleformat{\section}{\bfseries\fontsize{14pt}{14pt}\selectfont}{\thesection.}{1em}{}
\titlespacing{\section}{0pt}{6pt}{6pt}

% Настройка абзацев
\setlength{\parindent}{1.25cm} % Абзацный отступ
\setlength{\parskip}{6pt} % Интервал между абзацами

% Оформление цитат
\newenvironment{quoteformat}{\bfseries}{}

\begin{document}
	\begin{titlepage}
		\begin{center}
			% Шапка
			{\large \textbf{Министерство науки и высшего образования \\ Российской Федерации}}
			
			{\large\textbf{Федеральное государственное автономное образовательное учреждение высшего образования}}
			
			{\large \textbf{«Южно-Уральский государственный университет (НИУ)»}}
			
			{\large \textbf{Высшая школа электроники и компьютерных наук}}
			
			{\large \textbf{Кафедра системного программирования}\\[2cm]
			}
			% Тип работы
			\textbf{ОТЧЕТ}\\[0.2cm]
			о выполнении практического задания №2\\[0.2cm]
			по дисциплине\\[0.2cm]
			«Структуры и алгоритмы обработки данных»\\[0.2cm]
			\textbf{Вариант 5}\\[3cm]
		\end{center}
		
		\begin{flushright}
			Проверил:\\[0.2cm]
			ст. преподаватель кафедры СП\\[0.2cm]
			\textbf{Петрова Л.Н.}\\[1cm]
						
			Выполнил:\\[0.2cm]
			Студент группы КЭз-391\\[0.2cm]
			\textbf{Галиулин Р.Р.}\\[0.2cm]
			
		\end{flushright}
		\vfill{}
		
		\begin{center}
			Челябинск \\ 2025
		\end{center}
	\end{titlepage}
	\newpage
	
	\tableofcontents
	
	\setcounter{page}{2}
	\newpage
	%\maketitle
	\section{Описание задачи}
	\textbf{Задание №1: Односвязанный список}
	
	Информационная запись о файле содержит следующие поля: каталог, имя файла, 
	расширение, дата и время создания, атрибуты «только для чтения», «скрытый», 
	«системный», количество выделенных секторов (размер сектора принять равным 512 
	байтам). Поиск и сортировка — по каталогу, дате создания файла. 
	Выяснить, поместится ли файл на носитель с некоторым количеством секторов.
	
	\textbf{Входные данные:}
	
	\begin{itemize}
		\item Размер носителя - целое положительное число больше нуля
		\item Параметр поиска - буква латинского алфавита: \textbf{с} или \textbf{d}
		\item Каталог для поиска - строка
		\item Дата создания для поиска - строка в формате DD.MM.YYYY, где DD - день, MM -месяц, YYYY - год 
	\end{itemize}
	\textit{Все данные вводятся с помощью стандартного потока вывода.}	

	\textbf{Выходные данные:}
	
	\begin{itemize}
		\item Список файлов, содержащий сведения о файлах с данными указанными в задании. Выводится вначале, после сортировок и поиска. Формат буквенно-числовой. 
	\end{itemize}
	
	\textit{Все данные выводятся с помощью стандартного потока вывода.}	
		
	\textbf{Задание №2: Стек}
	
	Сформировать стек из N чисел. Извлечь элементы из стека, найти их сумму и произведение. Результат поместить в стек.
	
	\textbf{Входные данные:}
	
	\begin{itemize}
		\item Размер стека (кол-во элементов) - целое положительное число больше нуля
		\item Элементы стека - целые числа
	\end{itemize}
	
	\textit{Все данные вводятся с помощью стандартного потока вывода.}		
	
	\textbf{Выходные данные:}
	
	\begin{itemize}
		\item Начальный стек, список элементов стека - целых чисел
		\item Сумма числе элементов стека, целое число
		\item Произведение числе элементов стека, целое число
		\item Конечный стек, содержащие элементы начального стека, также сумму и произведение в качестве элементов - целые числа
	\end{itemize}
	
	
	\textit{Все данные выводятся с помощью стандартного потока вывода.}	
	
	\newpage
	\section{Листинги программ}
	Язык программирования: C++ 14. Среда разработки: Ubuntu 24.10, gcc 14.2.0, nvim
	
	При реализации задачи намерено, в учебных целях, не использованы контейнеры стандартной библиотеки \textbf{forward\_list} и его методы 
	\lstinputlisting[caption=Задание 1: Односвязанный список,label={lst:listing-cpp}, language=C++, firstline=1]{app3.cpp}
	
	
	\lstinputlisting[caption=Задание 2: Стек,label={lst:listing-cpp}, language=C++]{app4.cpp}
	
	\newpage	
	\section{Контрольные тесты}
	
	\textbf{Задание №1: Односвязанный список}
	
	\textit{\textbf{Пример работы}}
	
Файлы до сортировки:
DIR: /home/user/docs, NAME: file1, EXT: .txt, C\_DATE: 10.01.2023, C\_TIME: 12:00, R: 1, H: 0, S: 0, SIZE(512б): 50 \newline
...
\newline
DIR: /home/user/videos, NAME: video1, EXT: .mp4, C\_DATE: 01.01.2023, C\_TIME: 09:00, R: 1, H: 1, S: 0, SIZE(512б): 300
\newline
Файлы после сортировки по каталогу: \newline
DIR: /home/user/docs, NAME: file2, EXT: .docx, C\_DATE: 09.01.2023, C\_TIME: 10:00, R: 0, H: 1, S: 0, SIZE(512б): 150
\newline
...
\newline
DIR: /home/user/videos, NAME: video1, EXT: .mp4, C\_DATE: 01.01.2023, C\_TIME: 09:00, R: 1, H: 1, S: 0, SIZE(512б): 300
\newline
Файлы после сортировки по дате создания: \newline
DIR: /home/user/music, NAME: song1, EXT: .mp3, C\_DATE: 05.01.2022, C\_TIME: 18:30, R: 0, H: 0, S: 1, SIZE(512б): 200
\newline
...
\newline
DIR: /home/user/docs, NAME: file1, EXT: .txt, C\_DATE: 10.01.2023, C\_TIME: 12:00, R: 1, H: 0, S: 0, SIZE(512б): 50
\newline
Введите размер носителя в секторах (512 байт): \textbf{150} \newline
Каталог: /home/user/music, Имя файла: song1, Размер в секторах (512б): 200 - Не поместится на носитель
\newline
...
\newline
Каталог: /home/user/docs, Имя файла: file1, Размер в секторах (512б): 50 - Поместится на носитель
\newline
По какому параметру искать файлы? (d - по каталогу, c - по дате создания): \textbf{c} \newline
Введите дату создания для поиска (формат DD.MM.YYYY): \textbf{09.01.2023} \newline
DIR: /home/user/docs, NAME: file2, EXT: .docx, C\_DATE: 09.01.2023, C\_TIME: 10:00, R: 0, H: 1, S: 0, SIZE(512б): 150 \newline

	
	\renewcommand{\arraystretch}{1.5} % Увеличиваем высоту строк
	\subsection{Задание №2: Строка}
	\begin{table}[ht]
		
		\centering
		\begin{tabularx}{\textwidth}{|X|X|}
			\hline
			\textbf{Исходные данные} & \textbf{Результат} \\ \hline
			5 \newline 1 2 3 4 5 & 
			Начальный стек: 5 4 3 2 1 \newline
			Сумма: 15 \newline
			Произведение: 120 \newline
			Конечный стек: 120 15 5 4 3 2 1 \newline
			\\ \hline
			3 \newline 2 3 4  &
			Начальный стек: 4 3 2 \newline
			Сумма: 9 \newline
			Произведение: 24 \newline
			Конечный стек: 24 9 5 4 3 2 1 \newline
			\\ \hline
			
		\end{tabularx}
		\caption{Таблица с результатами контрольных тестов Задания №2}
	\end{table}
	
	
	\section{Контрольные вопросы}
	\subsection{Что такое ссылка?}
	
	Ссылка — это особый тип переменной, которая, аналогично указателю, указывает на область памяти. Однако, в отличие от указателя, ссылка должна быть инициализирована в месте объявления, не может быть переназначена на другой объект и не может ссылаться на неопределённую или несуществующую область памяти.
	Безопаснее чем указатель и позволяет также эффективно передавать переменные или структуры данных например в вызов функции. 
	
	\subsection{Линейный (односвязный) список — что это?}
	
	Односвязный список — это динамическая структура данных, представляющая собой последовательность, каждый элемент которой содержит данные и ссылку (или указатель) на следующий элемент (кроме последнего, который указывает на nullptr). Такая структура позволяет эффективно добавлять или удалять элементы, но имеет низкую эффективность при частом доступе к произвольным элементам, так как для этого требуется последовательный перебор от начала списка.
	
	\subsection{Какие операции можно выполнять с односвязным списком?}
	
	С односвязным списком возможно выполнять следующие операции:  
	
	\begin{itemize}
		\item \textbf{Добавление элементов}  
		\begin{itemize}
			\item В начало или конец последовательности (без изменения указателей других элементов).  
			\item Вставка в середину списка требует изменения указателя предыдущего элемента.  
		\end{itemize}
		
		\item \textbf{Удаление элементов}  
		\begin{itemize}
			\item Удаление первого элемента выполняется путём обновления головы списка.  
			\item Удаление последнего элемента требует обхода списка до предпоследнего узла.  
			\item Удаление элемента из середины требует изменения указателя предыдущего узла.  
		\end{itemize}
		
		\item \textbf{Обращение всего списка}  
		\begin{itemize}
			\item Меняется направление указателей в каждом узле так, чтобы каждый элемент указывал на предыдущий, а первый элемент становился последним.  
		\end{itemize}
		
		\item \textbf{Обход списка}  
		\begin{itemize}
			\item Проход по всем узлам для выполнения операций, таких как поиск, изменение или обработка элементов.  
		\end{itemize}
	\end{itemize}

	
	\subsection{Понятие стека. Операции, выполняемые над стеком.}
	
	Стек это структура данных реализующая прицип LIFO (последним пришел, первым вышел).
	
	Основными операциями являются:
	
	\begin{itemize}
		\item \textbf{Push} - добавление элемента "сверху"
		\item \textbf{Pop} - извлечение элемента "сверху"
	\end{itemize}
	
	Также существуют, но не во всех реализациях, дополнительные операции:
	
	\begin{itemize}
		\item \textbf{Top} (или Peek) - просмотр элемента без извлечения
		\item Проверка на пустоту, наличия хотя-бы одного элемента.
		\item \textbf{Size} - размер стека
	\end{itemize}
	
	
	\subsection{Представление стека с помощью массива. Выполнение основных операций.}
	
	Логичнее всего реализовать на основе динамического массива и работать с нём только через функции реализующие на ним операции указанные в пункте выше.
	
	\begin{lstlisting} [language=C]
		class Stack {
			private:
			int* array;  
			int top;
			int capacity;
			
			void resize() {
				capacity *= 2;
				int* new_array = new int[capacity];
				for (int i = 0; i < top; ++i) {
					new_array[i] = array[i];
				}
				delete[] arr;
				arr = new_array;
			}
			
			public:
			Stack() {
				capacity = 10;
				array = new int[capacity];
				top = 0;
			}
			
			~Stack() {
				delete[] array;
			}
			

			void push(int value) {
				if (top == capacity) {
					resize();
				}
				array[top++] = value;
			}
		
			int pop() {
				if (top == 0) {
					std::cout << "Стек пуст!" << std::endl;
					return -1;
				}
				return array[--top];
			}
			
			int peek() {
				if (top == 0) {
					std::cout << "Стек пуст!" << std::endl;
					return -1;
				}
				return array[top - 1];
			}
			

			bool isEmpty() {
				return top == 0;
			}
			
			int size() {
				return top;
			}
		};
		
	\end{lstlisting}
	 
	
	
	
\end{document}
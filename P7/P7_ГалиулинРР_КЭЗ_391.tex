\documentclass[a4paper,12pt]{article}

% Подключение необходимых пакетов
\usepackage[left=30mm,right=10mm,top=2cm,bottom=2cm]{geometry} % Поля 
\usepackage{amsmath,amsthm,amssymb}
\usepackage{mathtext}
\usepackage[T1,T2A]{fontenc}
\usepackage[utf8]{inputenc}
\usepackage[english,russian]{babel}
\usepackage{graphicx}

\usepackage{setspace} % Межстрочные интервалы
\usepackage{indentfirst} % Отступ у первого абзаца
\usepackage{titlesec} % Настройка заголовков
\usepackage{csquotes} % Для цитат
\usepackage{hyperref} % Для гиперссылок
\usepackage{amsmath}
\usepackage{amssymb}
\usepackage{xypic}
\usepackage{tikz}

\usepackage{listings}
\lstset{inputencoding=utf8x, extendedchars=\true}

% Define a custom color
\definecolor{backcolour}{rgb}{0.95,0.95,0.92}
\definecolor{codegreen}{rgb}{0,0.6,0}

% Define a custom style
\lstdefinestyle{myStyle}{
	backgroundcolor=\color{backcolour},   
	commentstyle=\color{codegreen},
	basicstyle=\ttfamily\footnotesize,
	breakatwhitespace=false,         
	breaklines=true,                 
	keepspaces=true,                 
	numbers=left,       
	numbersep=5pt,                  
	showspaces=false,                
	showstringspaces=false,
	showtabs=false,                  
	tabsize=2,
	basicstyle=\footnotesize\ttfamily,
	keywordstyle=\bfseries\color{green!40!black},
	commentstyle=\itshape\color{purple!40!black},
	identifierstyle=\color{blue},
	backgroundcolor=\color{gray!10!white},
}

% Use \lstset to make myStyle the global default
\lstset{style=myStyle}


\usetikzlibrary{positioning}

% Форматирование заголовков разделов
\titleformat{\section}{\bfseries\fontsize{14pt}{14pt}\selectfont}{\thesection.}{1em}{}
\titlespacing{\section}{0pt}{6pt}{6pt}

% Настройка абзацев
\setlength{\parindent}{1.25cm} % Абзацный отступ
\setlength{\parskip}{6pt} % Интервал между абзацами

% Оформление цитат
\newenvironment{quoteformat}{\bfseries}{}


%opening
\title{\textbf{Контрольное мероприятие №1} \\
	Контрольная работа по микроэкономике }
\author{Галиулин Ринат Рамильевич, КЭз-391}
\date{}

\begin{document}
	\begin{titlepage}
		\begin{center}
			% Шапка
			{\large \textbf{Министерство науки и высшего образования \\ Российской Федерации}}
			
			{\large\textbf{Федеральное государственное автономное образовательное учреждение высшего образования}}
			
			{\large \textbf{«Южно-Уральский государственный университет (НИУ)»}}
			
			{\large \textbf{Высшая школа электроники и компьютерных наук}}
			
			{\large \textbf{Кафедра системного программирования}\\[2cm]
			}
			% Тип работы
			\textbf{ОТЧЕТ}\\[0.2cm]
			о выполнении дополнительного практического задания №3\\[0.2cm]
			по дисциплине\\[0.2cm]
			«Структуры и алгоритмы обработки данных»\\[0.2cm]
			\textbf{Вариант 5}\\[3cm]
		\end{center}
		
		\begin{flushright}
			Проверил:\\[0.2cm]
			ст. преподаватель кафедры СП\\[0.2cm]
			\textbf{Петрова Л.Н.}\\[1cm]
						
			Выполнил:\\[0.2cm]
			Студент группы КЭз-391\\[0.2cm]
			\textbf{Галиулин Р.Р.}\\[0.2cm]
			
		\end{flushright}
		\vfill{}
		
		\begin{center}
			Челябинск \\ 2025
		\end{center}
	\end{titlepage}
	\newpage
	
	\tableofcontents
	
	\setcounter{page}{2}
	\newpage
	%\maketitle
	\section{Описание задачи}
	\textbf{Задача: Рекурсия с возвратом / Ход конём}
	
	Найти маршрут обхода конем шахматной доски, заданных размеров, из заданного начального положения коня.
	
	\textbf{Входные данные} (файл input.txt):
	
	M, N - размеры шахматной доски.
	X, Y - начальные координаты расположения коня.
	
	\textbf{Выходные данные} (файл output.txt): Напечатать номер хода в каждой ячейки поля, если маршрут существует и "Маршрут не существует" в противном случае.
	
	Пример:
	
	Входные данные: M=10, N=10, X=1, Y=1.
	
	Выходные данные:

	\begin{tabular}{*{10}{r}}
		28 & 63 & 12 & 45 & 26 & 61 & 10 & 43 & 24 & 59 \\
		13 & 46 & 27 & 62 & 11 & 44 & 25 & 60 & 9 & 42 \\
		64 & 29 & 94 & 77 & 86 & 69 & 92 & 79 & 58 & 23 \\
		47 & 14 & 85 & 68 & 93 & 78 & 87 & 70 & 41 & 8 \\
		30 & 65 & 76 & 95 & 84 & 91 & 80 & 97 & 22 & 57 \\
		15 & 48 & 67 & 100 & 75 & 96 & 71 & 88 & 7 & 40 \\
		66 & 31 & 50 & 83 & 90 & 73 & 98 & 81 & 56 & 21 \\
		49 & 16 & 33 & 74 & 99 & 82 & 89 & 72 & 39 & 6 \\
		32 & 51 & 2 & 35 & 18 & 53 & 4 & 37 & 20 & 55 \\
		1 & 34 & 17 & 52 & 3 & 36 & 19 & 54 & 5 & 38 \\
	\end{tabular}
	
	Для решения задачи был применён эвристический метод Варнсдорфа - поиска пути с минимальным количеством дальнейших шагов и избегания потенциально тупиковых ситуации, вроде угловых клеток.
	
	\newpage
	\section{Листинг программы}
	Язык программирования: C++ 14. Среда разработки: Ubuntu 24.10, gcc 14.2.0, nvim
	\lstinputlisting[caption=Задача: Ход конем,label={lst:listing-cpp}, language=C++, firstline=1]{app10.cpp}
	
	\newpage
	\section{Контрольные тесты}
	
	\textbf{Тест 1}
	
	\textit{Входные данные:} M=6, N=6, X=3, Y=3.
	
	\textit{Выходные данные:} \newline
	\begin{center}
	\begin{tabular}{*{6}{r}}
		31 &  2 & 21 & 16 & 29 &  8 \\ 
		22 & 15 & 30 &  9 & 20 & 17 \\
		3 & 32 &  1 & 18 &  7 & 28 \\
		14 & 23 & 36 & 27 & 10 & 19 \\
		33 &  4 & 25 & 12 & 35 &  6 \\
		24 & 13 & 34 &  5 & 26 & 11 \\
	\end{tabular}
	\end{center}
	
	\textbf{Тест 2}
	
	\textit{Входные данные:} M=4, N=4, X=1, Y=1.
	
	\textit{Выходные данные:} \newline
	\begin{center}
		Маршрут не существует
	\end{center}
	
	\textbf{Тест 3}
	
	\textit{Входные данные:} M=5, N=5, X=1, Y=1.
	
	\textit{Выходные данные:} \newline
	\begin{center}
		\begin{tabular}{*{5}{r}}
			1 & 22 & 11 & 16 &  7 \\
			12 & 17 &  8 & 21 & 10 \\
			25 &  2 & 23 &  6 & 15 \\
			18 & 13 &  4 &  9 & 20 \\
			3 & 24 & 19 & 14 &  5 \\
		\end{tabular}
	\end{center}
	
	\textbf{Тест 4}
	
	\textit{Входные данные:} M=3, N=3, X=1, Y=1.
	
	\textit{Выходные данные:} \newline
	\begin{center}
		Маршрут не существует
	\end{center}
	
	\textbf{Тест 3}
	

	
\end{document}
\documentclass[a4paper,12pt]{article}

% Подключение необходимых пакетов
\usepackage[left=30mm,right=10mm,top=2cm,bottom=2cm]{geometry} % Поля 
\usepackage{amsmath,amsthm,amssymb}
\usepackage{mathtext}
\usepackage[T1,T2A]{fontenc}
\usepackage[utf8]{inputenc}
\usepackage[english,russian]{babel}
\usepackage{graphicx}
\usepackage{tabularx} % Пакет для таблиц с автоматической шириной столбцов

\usepackage{setspace} % Межстрочные интервалы
\usepackage{indentfirst} % Отступ у первого абзаца
\usepackage{titlesec} % Настройка заголовков
\usepackage{csquotes} % Для цитат
\usepackage{hyperref} % Для гиперссылок
\usepackage{amsmath}
\usepackage{amssymb}
\usepackage{xypic}
\usepackage{tikz}
\usepackage{tabularx} % Пакет для таблиц с автоматической шириной столбцов

\usepackage{listings}
\lstset{inputencoding=utf8x, extendedchars=\true}

% Define a custom color
\definecolor{backcolour}{rgb}{0.95,0.95,0.92}
\definecolor{codegreen}{rgb}{0,0.6,0}

% Define a custom style
\lstdefinestyle{myStyle}{
	backgroundcolor=\color{backcolour},   
	commentstyle=\color{codegreen},
	basicstyle=\ttfamily\footnotesize,
	breakatwhitespace=false,         
	breaklines=true,                 
	keepspaces=true,                 
	numbers=left,       
	numbersep=5pt,                  
	showspaces=false,                
	showstringspaces=false,
	showtabs=false,                  
	tabsize=2,
	basicstyle=\footnotesize\ttfamily,
	keywordstyle=\bfseries\color{green!40!black},
	commentstyle=\itshape\color{purple!40!black},
	identifierstyle=\color{blue},
	backgroundcolor=\color{gray!10!white},
}

% Use \lstset to make myStyle the global default
\lstset{style=myStyle}


\usetikzlibrary{positioning}

% Форматирование заголовков разделов
\titleformat{\section}{\bfseries\fontsize{14pt}{14pt}\selectfont}{\thesection.}{1em}{}
\titlespacing{\section}{0pt}{6pt}{6pt}

% Настройка абзацев
\setlength{\parindent}{1.25cm} % Абзацный отступ
\setlength{\parskip}{6pt} % Интервал между абзацами

% Оформление цитат
\newenvironment{quoteformat}{\bfseries}{}


%opening
\title{\textbf{Контрольное мероприятие №1} \\
	Контрольная работа по микроэкономике }
\author{Галиулин Ринат Рамильевич, КЭз-391}
\date{}

\begin{document}
	\begin{titlepage}
		\begin{center}
			% Шапка
			{\large \textbf{Министерство науки и высшего образования \\ Российской Федерации}}
			
			{\large\textbf{Федеральное государственное автономное образовательное учреждение высшего образования}}
			
			{\large \textbf{«Южно-Уральский государственный университет (НИУ)»}}
			
			{\large \textbf{Высшая школа электроники и компьютерных наук}}
			
			{\large \textbf{Кафедра системного программирования}\\[2cm]
			}
			% Тип работы
			\textbf{ОТЧЕТ}\\[0.2cm]
			о выполнении практического задания №4\\[0.2cm]
			по дисциплине\\[0.2cm]
			«Структуры и алгоритмы обработки данных»\\[0.2cm]
			\textbf{Вариант 5}\\[3cm]
		\end{center}
		
		\begin{flushright}
			Проверил:\\[0.2cm]
			ст. преподаватель кафедры СП\\[0.2cm]
			\textbf{Петрова Л.Н.}\\[1cm]
						
			Выполнил:\\[0.2cm]
			Студент группы КЭз-391\\[0.2cm]
			\textbf{Галиулин Р.Р.}\\[0.2cm]
			
		\end{flushright}
		\vfill{}
		
		\begin{center}
			Челябинск \\ 2025
		\end{center}
	\end{titlepage}
	\newpage
	
	\tableofcontents
	
	\setcounter{page}{2}
	\newpage
	%\maketitle
	\section{Описание задачи}
	\textbf{Задание: Алгоритмы сортировки данных}
	
	Отсортировать одномерный массив вещественных чисел размерности N, 
	применив сортировки бинарным включением и шейкерную.

	\textbf{Входные данные:}
	\begin{itemize}
		\item Размер массива n, целое число больше нуля
		\item Элементы массива, вещественные числа
	\end{itemize}
	
	\textit{Все данные вводятся через стандартный поток ввода}
	
	\textbf{Выходные данные:}
	
	\begin{itemize}
		\item Начальные элементы массива, введённой последовательности
		\item Отсортированный массив элементов, алгоритмом сортировки бинарным включением
		\item Отсортированный массив элементов, алгоритмом шейкирной сортировки
		\item Время выполнения каждого из алгоритмов
	\end{itemize}
	
	\textit{Данные выводятся через стандартный поток вывода}
	
	\newpage
	\section{Листинги программ}
	Язык программирования: C++ 14. Среда разработки: Ubuntu 24.10 (6.11.0-13-generic), gcc 14.2.0, neovim
	\lstinputlisting[caption=Задание: Алгоритмы сортировки данных ,label={lst:listing-cpp}, language=C++, firstline=1]{app7.cpp}
		
	\newpage
	\section{Контрольные тесты}
	
	\subsection{Задание №1: Сортировка}
	\begin{table}[ht]
		\centering
		\begin{tabularx}{\textwidth}{|X|X|}
			\hline
			\textbf{Исходные данные} & \textbf{Результат} \\ \hline
			10 \newline 45  24  1  56  8  9  47  14  20  56  & 
			Массив после сортировки бинарными включениями: 1 8 9 14 20 24 45 47 56 56 \newline
			Время выполнения бинарной сортировки: 3.748e-06 секунд \newline
			Массив после шейкерной сортировки: 1 8 9 14 20 24 45 47 56 56 \newline
			Время выполнения шейкерной сортировки: 1.009e-06 секунд \newline
			    \\ \hline
			5 \newline  8  4  6  2  0 &
			Массив после сортировки бинарными включениями: 0 2 4 6 8 \newline
			Время выполнения бинарной сортировки: 2.096e-06 секунд \newline
			Массив после шейкерной сортировки: 0 2 4 6 8 \newline
			Время выполнения шейкерной сортировки: 5.13e-07 секунд \newline
			    \\ \hline
		\end{tabularx}
		\caption{Таблица с результатами контрольных тестов Задания №1}
	\end{table}
	
	\newpage
	\section{Контрольные вопросы}
	\subsection{Дайте определение понятию «сортировка»}
	
	Это упорядочение элементов множества в некотором порядке, например по возрастанию значений элементов, с целью оптимизации работы с ним, в случае поиска элемента по значению, или анализу (например нахождения медианного значения). Используется в качестве подготовки для анализа данных, поиска и многих других алгоритмов.     
	
	\subsection{Назовите отличия между внутренней и внешней сортировкой}
	
	Внешняя сортировка применяется, когда объём данных, который необходимо отсортировать, превышает объём доступной оперативной памяти. В этом случае для хранения данных используются внешние устройства, такие как жёсткий диск или SSD.
	
	Из-за необходимости чтения и записи данных на внешние носители внешняя сортировка значительно медленнее, чем внутренняя. Однако она позволяет обрабатывать очень большие объёмы данных, недоступные для внутренней сортировки.
	
	Для внешней сортировки требуются специализированные алгоритмы, такие как внешняя сортировка слиянием или многопроходные методы. Эти алгоритмы минимизируют количество операций ввода-вывода, чтобы повысить эффективность процесса.
	
	\subsection{Назовите принципы действия сортировки выбором.}
	
	Основана на последовательном выборе минимального (или максимального) элемента из неотсортированной части массива и его перемещении в начало (или конец) отсортированной части. Для примера с максимальным элементом:
	
	\begin{enumerate}

		\item Условно разделяем массив на две части: не отсортированную и пустую отсортированную часть
		
		\item Находим максимальный элемент в неотсортированной части.
		
		\item Перемещаем найденный элемент, меняя его местами с крайним элементом текущей неотсортированной части.
		
		\item Повторяем процесс, постепенно уменьшая размер неотсортированной части.
	
	\end{enumerate}

	Алгоритм прост в реализации и не требует дополнительной памяти, так как работает "на месте". Однако его временная сложность $O(n^{2})$ делает его неэффективным для работы с большими массивами.

	\subsection{Назовите принципы действия обменной сортировки}
	
	Алгоритм обменной сортировки (он же пузырьковая сортировка) основывается на последовательном сравнении и обмене соседних элементов, если они расположены не в порядке (по убыванию или возрастанию). На первой итерации самый большой (или самый маленький) элемент "всплывает" и перемещается в конец последовательности. На следующей итерации неотсортированная часть последовательности уменьшается на один элемент. Процесс повторяется, пока размер неотсортированной части не станет равен нулю.
	
	Однако временная сложность алгоритма $O(n^{2})$ в худшем и среднем случае делает его неэффективным для работы с большими массивами. Тем не менее, в лучшем случае (при изначально отсортированном массиве) его сложность может быть снижена до $O(n)$.
	
	\subsection{Назовите принципы действия шейкерной сортировки}

	Принцип основан на обменной (пузырьковой) сортировке, но алгоритм проходит последовательность в обоих направлениях, что делает его немного эффективнее:
	\begin{enumerate}	
	\item Сначала в прямом порядке, перемещая максимальный элемент в конец последовательности.
	\item Затем в обратном порядке, перемещая минимальный элемент в начало.
	\end{enumerate}
	После каждой итерации область, требующая сортировки, уменьшается на два элемента: максимальный и минимальный элементы занимают свои окончательные позиции. Однако это не решает проблему низкой эффективности, так как временная сложность остаётся $O(n^{2})$.

	\subsection{Назовите принципы действия сортировки вставками}
	
	Сортировка вставками упорядочивает массив, последовательно добавляя элементы из неотсортированной части в правильное место в отсортированной части. Последовательность делится на отсортированную (изначально включает только первый элемент) и неотсортированную части. Затем второй и последующие элементы сравниваются с элементами отсортированной части. Все элементы, которые больше (или меньше) текущего, сдвигаются вправо, после чего текущий элемент вставляется на своё место.
	
\end{document}
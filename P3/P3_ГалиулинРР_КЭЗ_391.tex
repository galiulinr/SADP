\documentclass[a4paper,12pt]{article}

% Подключение необходимых пакетов
\usepackage[left=30mm,right=10mm,top=2cm,bottom=2cm]{geometry} % Поля 
\usepackage{amsmath,amsthm,amssymb}
\usepackage{mathtext}
\usepackage[T1,T2A]{fontenc}
\usepackage[utf8]{inputenc}
\usepackage[english,russian]{babel}
\usepackage{graphicx}
\usepackage{tabularx} % Пакет для таблиц с автоматической шириной столбцов

\usepackage{setspace} % Межстрочные интервалы
\usepackage{indentfirst} % Отступ у первого абзаца
\usepackage{titlesec} % Настройка заголовков
\usepackage{csquotes} % Для цитат
\usepackage{hyperref} % Для гиперссылок
\usepackage{amsmath}
\usepackage{amssymb}
\usepackage{xypic}
\usepackage{tikz}

\usepackage{listings}
\lstset{inputencoding=utf8x, extendedchars=\true}

% Define a custom color
\definecolor{backcolour}{rgb}{0.95,0.95,0.92}
\definecolor{codegreen}{rgb}{0,0.6,0}

% Define a custom style
\lstdefinestyle{myStyle}{
	backgroundcolor=\color{backcolour},   
	commentstyle=\color{codegreen},
	basicstyle=\ttfamily\footnotesize,
	breakatwhitespace=false,         
	breaklines=true,                 
	keepspaces=true,                 
	numbers=left,       
	numbersep=5pt,                  
	showspaces=false,                
	showstringspaces=false,
	showtabs=false,                  
	tabsize=2,
	basicstyle=\footnotesize\ttfamily,
	keywordstyle=\bfseries\color{green!40!black},
	commentstyle=\itshape\color{purple!40!black},
	identifierstyle=\color{blue},
	backgroundcolor=\color{gray!10!white},
}

% Use \lstset to make myStyle the global default
\lstset{style=myStyle}


\usetikzlibrary{positioning}

% Форматирование заголовков разделов
\titleformat{\section}{\bfseries\fontsize{14pt}{14pt}\selectfont}{\thesection.}{1em}{}
\titlespacing{\section}{0pt}{6pt}{6pt}

% Настройка абзацев
\setlength{\parindent}{1.25cm} % Абзацный отступ
\setlength{\parskip}{6pt} % Интервал между абзацами

% Оформление цитат
\newenvironment{quoteformat}{\bfseries}{}


%opening
\title{\textbf{Контрольное мероприятие №3} \\
	Контрольная работа по микроэкономике }
\author{Галиулин Ринат Рамильевич, КЭз-391}
\date{}

\begin{document}
	\begin{titlepage}
		\begin{center}
			% Шапка
			{\large \textbf{Министерство науки и высшего образования \\ Российской Федерации}}
			
			{\large\textbf{Федеральное государственное автономное образовательное учреждение высшего образования}}
			
			{\large \textbf{«Южно-Уральский государственный университет (НИУ)»}}
			
			{\large \textbf{Высшая школа электроники и компьютерных наук}}
			
			{\large \textbf{Кафедра системного программирования}\\[2cm]
			}
			% Тип работы
			\textbf{ОТЧЕТ}\\[0.2cm]
			о выполнении практического задания №3\\[0.2cm]
			по дисциплине\\[0.2cm]
			«Структуры и алгоритмы обработки данных»\\[0.2cm]
			\textbf{Вариант 5}\\[3cm]
		\end{center}
		
		\begin{flushright}
			Проверил:\\[0.2cm]
			ст. преподаватель кафедры СП\\[0.2cm]
			\textbf{Петрова Л.Н.}\\[1cm]
						
			Выполнил:\\[0.2cm]
			Студент группы КЭз-391\\[0.2cm]
			\textbf{Галиулин Р.Р.}\\[0.2cm]
			
		\end{flushright}
		\vfill{}
		
		\begin{center}
			Челябинск \\ 2025
		\end{center}
	\end{titlepage}
	\newpage
	
	\tableofcontents
	
	\setcounter{page}{2}
	\newpage
	%\maketitle
	\section{Описание задачи}
	\textbf{Задание №1: Множество}
	
	Составить программу подсчёта общего количества цифр и знаков «+»,«-»,«*» 
	в строкe, введённой с клавиатуры.
	
	\textbf{Входные данные}
	
	\begin{itemize}
		\item Строка, содержащая число-буквенные символы
	\end{itemize}
	
	\textit{Все данные вводятся с помощью стандартного потока вывода.}		

	\textbf{Выходные данные}

	\begin{itemize}
		\item Количество символов «+»,«-»,«*» и цифр выведенных во входной строке. Целое положительное число или ноль.
	\end{itemize}

	\textit{Все данные вводятся с помощью стандартного потока вывода.}		

		
	\textbf{Задание №2: Очередь}
	
	Дана очередь вещественных чисел. Удалить из очереди числа из заданного 
	пользователем диапазона.

	\textbf{Входные данные}

	\begin{itemize}
		\item Очередь вещественных чисел
		\item Нижняя граница диапазона, вещественное число
		\item Верхняя граница диапазона, вещественное число 
	\end{itemize}

	\textit{Все данные вводятся с помощью стандартного потока вывода.}		

	\textbf{Выходные данные}

	\begin{itemize}
		\item Очередь вещественных чисел
	\end{itemize}

	\textit{Все данные вводятся с помощью стандартного потока вывода.}		
	
	\newpage	
	\section{Листинги программ}
	Язык программирования: C++ 14. Среда разработки: Ubuntu 24.10, gcc 14.2.0, nvim
	\lstinputlisting[caption=Задание 1: Множество,label={lst:listing-cpp}, language=C++, firstline=1]{app5.cpp}
	
	\lstinputlisting[caption=Задание 2: Очередь,label={lst:listing-cpp}, language=C++]{app6.cpp}
	
	\newpage
	\section{Контрольные тесты}
	
	\renewcommand{\arraystretch}{1.5} % Увеличиваем высоту строк
	\subsection{Задание №1: Множества}
	\begin{table}[ht]
		
		\centering
		\begin{tabularx}{\textwidth}{|X|X|}
			\hline
			\textbf{Исходные данные} & \textbf{Результат} \\ \hline
			A+4+5+6dfg56+-H41k12 & 14 \\ \hline
			0123456789g & 10 \\ \hline
			1gfhgjhg5 & 2 \\ \hline
			fhghjgj2hfhgjhg & 1 \\ \hline			
		\end{tabularx}
		\caption{Таблица с результатами контрольных тестов Задания №1}
	\end{table}
	
	\renewcommand{\arraystretch}{1.5} % Увеличиваем высоту строк
	\subsection{Задание №3: Очередь}
	\begin{table}[ht]
		
		\centering
		\begin{tabularx}{\textwidth}{|X|X|}
			\hline
			\textbf{Исходные данные} & \textbf{Результат} \\ \hline
			12 0 12 01 45 2 3 \newline 12 \newline 12 
			& 0 1 45 2 3 \\ \hline
			1 2 3 4 5 6 7 8 9 0 \newline 0 \newline 4 
			& 5 6 7 8 9 \\ \hline
			4 4 2 2 3 3 5 5 6 6 1 1 \newline 1 \newline 2
			& 4 4 3 3 5 5 6 6 \\ \hline		
		\end{tabularx}
		\caption{Таблица с результатами контрольных тестов Задания №2}
	\end{table}
	
	\newpage
	\section{Контрольные вопросы}
	\subsection{Дайте определение очереди}
	
	В отличии от стека очередь реализует принцип FIFO - первым пришёл, первым вышел, т.е. элементы добавляются в конец очереди, а извлекаются из начала. 
	
	\subsection{Где применяется очередь?}
	
	Где логически применим принцип FIFO. Например при буферизации данных используется в качестве временного хранилища данных поступающих с разной скоростью. Управлении задачами, когда важно выполнять все по очереди, например очередь печати принтера, очереди сообщений в операционных системах, обработка запросов к серверу, моделирование процессов В ряде алгоритмов работы с более сложными структурами данных, например графами.
	
	\subsection{Перечислите основные операции применяемые в очереди.}
	
	Основными операциями являются:
	
	\begin{itemize}
		\item \textbf{Push} - добавление элемента в конец
		\item \textbf{Pop} - извлечение элемента с начала
	\end{itemize}
	
	Также существуют, но не во всех реализациях, дополнительные операции:
	
	\begin{itemize}
		\item \textbf{Top} (или Peek) - просмотр начального элемента без извлечения
		\item Проверка на пустоту, наличия хотя-бы одного элемента.
		\item \textbf{Size} - размер стека
	\end{itemize}
	
	\subsection{Что такое множество?}
	
	В качестве структуры данных, являются отражением математического понятия, как неупорядоченную коллекцию различных элементов. Ключевым является что элементы - различны, т.е. множество не содержит в себе дубликаты.   
	
	\subsection{Дайте определение мощности множества.}
	
	Кол-во \textbf{различных} элементов множества. Если множество реализуется с соблюдением принципа различности элементов, то просто кол-во элементов множества.
	
	\subsection{Какие операции можно применять к множествам?}
	
	\begin{itemize}
		\item Добавление элемента в множество
		\item Удаление элемента из множества 
		\item Проверка на принадлежность некого значения множеству
		\item Объединение (аналог $Q \cup W$) - создание нового множества состоящего из элементов множеств подлежащих объединению (в std::set set\_union)
		\item Пересечение (аналог $Q \cap W$) - создание нового множества состоящих из элементов принадлежащих (состоящих) в множествах подлежащих пересечению (в std::set set\_intersection)
		\item Разность множеств (аналог $Q$-$W$) - создание нового множества элементы которого принадлежат Q но не принадлежат W (в std::set set\_difference)
		\item Проверка на подмножество (аналог $Q \subset W$) - логическая операция возвращающая истину если все элементы множества Q также содержаться в множестве W
	\end{itemize}
	
	
\end{document}